\documentclass{beamer}
\usetheme{CambridgeUS}
 
\usepackage{fontspec}
\setsansfont{楷体} % font name is case-sensitive
\author{南开大学ACM算法协会}
\begin{document}
\title{NKPC16题解} 
\frame{\titlepage}
\section{绝对精度一句话题意}
\begin{frame} {绝对精度}
  将一个无限循环小数化简为分数
\end{frame}
\section{绝对精度题解}
\begin{frame} {绝对精度}
\begin{itemize}
  \item 现在只截取循环部分的小数n=$0.q_1q_2q_3q_4...q_m...q_1q_2q_3q_4...q_m$
  \item 假设这个分数的循环节长度为$m$,所以满足方程
  $$
  10^m*n-\overline{q_1q_2q_3q_4...q_m}=n
  $$
  \item 然后用分数解方程就行,主要用Python,并且处理好行末$\backslash$n字符
\end{itemize}
\end{frame}


\section{无限空间一句话题意}
\begin{frame} {无限空间}
  计算$n$个平面在空间中最多能划分出几个无限大的空间
\end{frame}
\section{无限空间题解}
\begin{frame} {无限空间}
\begin{itemize}
  \item 二维找找规律,并且拓展一下到三维,通项公式为$$a_n=n*(n-1)+2$$
  \item 证明:
  \item 考虑$g(n)$为二维平面n条交点不同的直线形成的无限区域个数
  每一条新增直线最多产生两个无限区域,故$g(n)=2n$。
  \item 对于新的一个平面,
  原有的$n-1$个平面与之形成的交线在这个平面上
  最多形成$g(n-1)$个无限平面区域
  每个无限平面区域与新增的无限空间区域一一对应,
  $f(n)=f(n-1)+g(n-1)$
\end{itemize}
\end{frame}

\begin{frame} {无限空间}
  \begin{itemize}
    \item 平面区域到新增空间区域是双射的证明:
    \item 首先所有无限平面两侧的两个空间一定是无界的,
    因此肯定有一个新增的无限空间,因此是单射
    \item 任意一个有限平面区域,如果对应了两个无限区域,
    那么围成这个有限平面区域的几条线对应的平面,
    在空间中构成了一个棱柱侧面状,这一定不是最优解,
    所以最优解中新增的无限空间区域一定对应无限平面区域,
    因此是满射
  \end{itemize}
  \end{frame}


\section{FLAG之王2一句话题意}
\begin{frame} {FLAG之王2}
  给出一个$DAG$,问有多少种拓扑排序
\end{frame}
\section{FLAG之王2题解}
\begin{frame} {FLAG之王2}
\begin{itemize}
  \item 经典的拓扑计数问题
  \item $NPC$的问题,每次枚举入度为0的点拆下来,状压dp转移一下,
  拆的时候判断一下合法性,每个状态表示当前被拆剩下的点的方案数
  \item 时间复杂度$O(n*2^n)$
\end{itemize}
\end{frame}

\section{能赢吗一句话题意}
\begin{frame} {能赢吗}
  给出$n$堆麦子,每堆有$a_i$个麦粒,两人轮流取麦粒,谁先取完谁赢,
  后手可以先将一粒麦子从一个谷堆转移到另一个谷堆,他有必胜策略吗
\end{frame}
\section{能赢吗题解}
\begin{frame} {能赢吗}
\begin{itemize}
  \item $Nim$游戏魔改,就是转移之后能不能使得
  $a_1 \oplus a_2 \oplus ... \oplus a_n =0$
  \item 增加或减少1个谷粒会使得二进制的后$k$位发生变化,
  那移动一个谷粒会使n个谷堆数的异或二进制数的连续几个数位变化,
  b赢需要n个谷堆异或值二进制数中只能有符合条件的一串连续的1
\end{itemize}
\end{frame}

\section{虫洞一句话题意}
\begin{frame} {虫洞}
  给出一个图,每个顶点有物品,物品有质量,体积,价值,
  自己的背包有容量限制,通过边时有质量限制,
  从点1到点n能获得的最大价值。
\end{frame}
\section{虫洞题解}
\begin{frame} {虫洞}
\begin{itemize}
  \item 图论+背包的题一般都是指数时间复杂度
  \item 也是用状态压缩表示当前已经拿到的物品,
  然后用$dfs$或者$bfs$在图上遍历搜索,
  每次更新检测是否合法就行,注意重复状态的检查和自己的大常数
  \item 时间复杂度$O(n*2^n)$
\end{itemize}
\end{frame}


\section{物资分配一句话题意}
\begin{frame} {物资分配}
  需要$n$个物资,有四种资源$A,B,C,D$可以拿,其中$C,D$需要拿偶数件,
  问有多少种方案
\end{frame}
\section{物资分配题解}
\begin{frame} {物资分配}
\begin{itemize}
  \item 由生成函数的思想,答案为母函数$x^n$项系数
  \item 
  4种物资里两种无限制,它们的对应函数为
  $$1+x+(x^2/2!)+…  =e^x$$
  其余两种个数限制为偶数,对应的函数为
  $$1+(x^2/2!)+(x^4/4!)+…  = (e^x+e^{-x})/2$$
  把它们对应的函数乘起来,得到本题的生成函数:
  $$\frac{e^{4x}+2e^{2x}+1}{4}$$
  答案为$x^n$的系数乘以$n!$,为
  $$\frac{4^n+2^{n+1}}{4}$$
  \item 注意高精度
\end{itemize}
\end{frame}


\section{魔法匹配一句话题意}
\begin{frame} {魔法匹配}
  给定一个模板串$s$,对每个查询串求$s$上有多少个等长的子串
  满足本题设定的匹配规则
\end{frame}
\section{魔法匹配题解}
\begin{frame} {魔法匹配}
\begin{itemize}
  \item 我们离线处理,对每种查询串的长度分别考虑,
  首先可以通过滑动窗$O(n)$求出模板串$s$每个该长度的哈希值,
  然后和每个该长度的查询串比较,得出答案。
  \item 设查询串长度总和为$L$,可以得出查询串长度的种类是$O(\sqrt{L})$
  \item 哈希比较根据不同的方法有$O(1)$和$O(\log{n})$
  \item 总时间复杂度$O(n*\sqrt{L}*\log{n})$
\end{itemize}
\end{frame}


\section{地下通道的规划一句话题意}
\begin{frame} {地下通道的规划}
  两个排列之间连线,所连接的数字之差的绝对值不能超过4,
  并且连线不能交叉,端点也不能相交
\end{frame}
\section{地下通道的规划题解}
\begin{frame} {地下通道的规划}
\begin{itemize}
  \item 先是二维$dp$
  \item $f[i][j]$表示$A$的前$i$和$B$的前$j$个点最大连接数量
  转移方程
  $$
    f[i][j]=max(
      f[i-1][j],
      f[i-1][j-1],
      f[i-1][j-1]+1\text{ if } |A_i-B_j|\le 4)
  $$
  \item 优化之后将$f[i][j]$定义为$A_i,B_j$之间有连线的连线数
  \item 然后转移公式类似,用线段树或树状数组处理一下即可
\end{itemize}
\end{frame}



\section{我要成为传奇小白一句话题意}
\begin{frame} {我要成为传奇小白}
  给定一系列字符串,可以将它们首尾相接拼起来,
  并且当$A$串的后缀和$B$串的前缀相同时,可以自行选择重叠长度
\end{frame}
\section{我要成为传奇小白题解}
\begin{frame} {我要成为传奇小白}
\begin{itemize}
  \item 考虑到要拼接的字符串数很小,可以用$dp$来求解,
  因为要每个串都选一次且只选一次,可以看作一个$TSP$问题,
  利用状压$dp$来求解
  \item 那么接下来的问题是怎么让拼接的长度最小且不包含负面效果的串,
  也就是我们需要知道任意两个串拼起来的花费是多少,
  知道了这项就可以直接进行$dp$求解
\end{itemize}
\end{frame}


\begin{frame} {我要成为传奇小白}
为了进行多字符串匹配,我们需要构建一个$AC$自动机,
要用所有给定的初始字符串与负面效果串构建自动机,
但是负面效果的串要做标记。而后在寻找两个串拼起来的花费时
利用$spfa$在构建好的$Trie$上求解即可
这时要注意把做过标记的负面效果匹配结果筛掉。
\end{frame}
\end{document}
